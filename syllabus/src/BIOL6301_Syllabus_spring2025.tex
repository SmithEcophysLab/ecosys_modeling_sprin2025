\documentclass[12pt, notitlepage]{article}   	% use "amsart" instead of "article" for AMSLaTeX format
\usepackage{geometry}                		% See geometry.pdf to learn the layout options. There are lots.
\geometry{a4paper}                   		% ... or a4paper or a5paper or ... 
%\geometry{landscape}                		% Activate for rotated page geometry
\usepackage[parfill]{parskip}    		% Activate to begin paragraphs with an empty line rather than an indent
\usepackage{graphicx}				% Use pdf, png, jpg, or eps§ with pdflatex; use eps in DVI mode
								% TeX will automatically convert eps --> pdf in pdflatex

\usepackage{hyperref}
		
%SetFonts

\usepackage[T1]{fontenc}
\usepackage[utf8]{inputenc}

\usepackage{tgbonum}

%SetFonts

\title{
	\textbf{
		BIOL 6301-029
	} \\
	\large Terrestrial Ecosystem Modeling \\
	\large Spring 2025
}

\date{\vspace{-5ex}}

\begin{document}

{\fontfamily{phv}\selectfont %select helvetica (code = phv)

\maketitle

\section{Course Description}
Students in this course will learn the principles of terrestrial ecosystem modeling.
This will include the core concepts behind systems thinking and model building.
The ultimate goal of the course will be develop a terrestrial ecosystem model, with
a focus on carbon and nutrient cycle dynamics. Through course-based model development,
students will learn the skills necessary to develop their own model and to understand
the workings of models developed by others. The course is primarily for biology
graduate students with a background in plant physiological and ecosystem ecology.

\subsection{Class Time and Location}
Tuesdays and Thursdays 12:30-13:50

Science Building Room 204

\subsection{Instructor}
Dr. Nick Smith \par
Experimental Sciences Building II (ESBII) Room 402D \par
806-834-7363 \par
nick.smith@ttu.edu \par

\subsection{Office Hours}
Thursdays 14:00-14:50 \par
Experimental Sciences Building II (ESBII) Room 402D \par

\subsection{Recommended Texts}
Climate Change and Terrestrial Ecosystem Modeling
by Bonan
\url{https://doi.org/10.1017/9781107339217}\par

Principles of Terrestrial Ecosystem Ecology (2nd Edition; 2011) 
by Chapin, Matson, and Vitousek \par
The book can be accessed from Springer here: 
\url{https://link.springer.com/book/10.1007/978-1-4419-9504-9}. Click on "Access this title on 
SpringerLink." It can also be accessed through the TTU library.

\section{Course Materials}
All course materials, including lecture slides, readings, activities, and code 
will be posted to a GitHub repository for the course.
The primary repository address is
\url{https://github.com/SmithEcophysLab/ecosys_modeling_sprin2025}.
The repository will include the syllabus and all other miscellaneous class materials as the semester
progresses. A README file will contain information on the repository, including
links to different sections at 
\url{https://github.com/SmithEcophysLab/ecosys_modeling_sprin2025/README.md}.

\section{Learning Objective}
This course will broadly focus on understanding the principles underlying the development of
process-based terrestrial ecosystem models. An emphasis will be placed on how to build these
models to understand the interactions between terrestrial ecosystems and drivers of global
change, including climate change, changes in atmospheric gas concentration, and eutrophication.
Course topics will be taught through the lens of classical and more contemporary 
primary and secondary literature. There will be a strong emphasis on hands-on modeling skill
development. Topics will be flexible and modified to match student interests where possible.

\section{Attendance Policy}
Attendance will not be taken, but is strongly recommended. 
In class activity points will only be granted if students are in class.
Makeups will not be granted.

\section{Course Assessment}
\subsection{\textit{Participation and Engagement}}
Being an active and engaged participant in the class will benefit your understanding
of material as well as your peers'. Examples include asking questions, providing feedback,
and facilitating discussion.

\subsection{\textit{Quizzes}}
Short quizzes will be given periodically to test student knowledge of core concepts and to
stimulate discussion. In some cases, quizzes will be developed and administered
by class weekly co-leads.

\subsection{\textit{Homework Assignments}}
Short homework assignments will be given each week based on the Tuesday lectures and
Thursday model development discussions. These will typically be assigned on Thursday and
due the following Tuesday.

\subsection{\textit{Weekly co-leads}}
Throughout the semester, students will be asked to co-lead on the week's discussion
topic with Dr. Smith. This will consist of leading a jigsaw-style discussion of a literature article on Tuesdays
and developing a quiz for students on Thursdays.
Students will be evaluated on their ability to respond accurately to their peers' questions
as well as their ability to summarize and generate discussion on the week's topic.

\subsection{\textit{Model module development}}
The primary semester project will be to develop a module for the class terrestrial ecosystem
model based on the student's interest. The module will be written in R and must be able to
run as a stand-alone module as well as in connection with the larger class model.
The student will be required to write a full description of the module to coincide with the
code in both README and manuscript style format.

\section{Grading}
Participation and Engagement: 25\% \par
Quizzes: 5\% \par
Homework Assignments: 20\% \par
Weekly co-lead: 15\% \par
Module proposal: 10\% \par
Module presentation: 5\% \par
Final module: 20\% \par

Grades will be made available on Blackboard. 
All grades posted at the end of the course will be final, 
unless an error has been made in their calculation.
Please contact Dr. Smith if you feel your grade has been calculated incorrectly.

\section{Grading Scale}
A: $\geq$ 90\% \par
B: 80 – 90\% \par
C: 70 – 80\% \par
D: 60 – 70\% \par
F: $\leq$ 59.9\% \par

\section{Missing In-class Activities}
Students will be required to be in class to receive in-class activity points. 
Please contact Dr. Smith if you plan to miss class for a university function 
\textit{prior to class}. If class is missed due to an illness, 
please let Dr. Smith know as soon as possible. Documentation will need to be provided
in order to be able to make up any missed work.

\section{Special Considerations}
Texas Tech Policies Concerning Academic Honesty, Special Accommodations for 
Students with Disabilities, Student Absences for Observance of Religious Holy Days, 
Accommodations for Pregnant Students, and other policies may be found at this link: 
\url{https://www.depts.ttu.edu/tlpdc/RequiredSyllabusStatements.php}.

\subsection{AI Use}
The use of generative AI tools (such as ChatGPT) is strictly prohibited in this course for any purpose.
Information gathered from AI cannot be used even with appropriate citation. Submission of AI-generated
content (i.e., information, text, or images) as your own work is a violation of academic integrity and may
result in referral to the Office of Student Conduct. Please contact your instructor if you have questions
regarding this course policy.

\section{Plagiarism Statement}
Texas Tech University expects students to “understand the principles of academic integrity 
and abide by them in all class and/or course work at the University” (OP 34.12.5). 
Plagiarism is a form of academic misconduct that involves (1) the representation of words, 
ideas, illustrations, structure, computer code, other expression, or media of another as 
one's own and/or failing to properly cite direct, paraphrased, or summarized materials; 
or (2) self-plagiarism, which involves the submission of the same academic work more than 
once without the prior permission of the instructor and/or failure to correctly cite 
previous work written by the same student. Please review Section B of the TTU 
Student Handbook for more information related to other forms of academic misconduct, 
and contact your instructor if you have questions about plagiarism or other 
academic concerns in your courses. To learn more about the importance of 
academic integrity and practical tips for avoiding plagiarism, explore the 
resources provided by the TTU Library and the School of Law.

\newpage

\section*{Schedule of Topics by Week}
01/13/2023 – Introductions, semester planning, and goals \par
01/20/2023 – Systems Thinking; building systems diagrams \par
01/27/2023 – Terrestrial Ecosystem C & N cycling; downloading and using R \par
02/03/2023 – Terrestrial Ecosystem Models; GitHub and the class model environment \par
02/10/2023 – 3 R's of Terrestrial Ecosystem Modeling; working collaboratively in GitHub \par
02/17/2023 – Integrating Ecology into Models; writing functions in R \par
02/24/2023 – Eco-evolutionary Optimality Theory; model repository structure \par
03/03/2023 – Model development: Carbon In \par
03/10/2023 – \textbf{Module proposal presentations} \par
03/17/2023 – NO CLASS \par
03/24/2023 – Model Development: Plant Carbon Allocation \par
03/31/2023 – Model Development: Plant Nutrient Demand and Acquisition \par
04/07/2023 - Model Development: Soil Carbon Cycling \par
04/14/2023 – \textbf{Module presentations} \par
04/21/2023 – \textbf{Module presentations}; \textbf{Final module due} \par
04/28/2023 – NO CLASS
05/06/2023 – NO CLASS

\section*{General Weekly Schedule}
Generally, each Tuesday will consist of a lecture and jigsaw-style paper discussion. Thursdays
will consist of a quiz, paper discussion, and class model development exercise.

} %end font selection

\end{document} 
